\chapter{Running a Node}
Launch a node with optional arguments:
\begin{verbatim}
cargo run -- --port <PORT> --node-name <NAME> --peers <PEER1,PEER2,...> --chain-dir <DIR> --network <mainnet|testnet>
\end{verbatim}
When the \texttt{tls} feature is enabled you can provide certificate paths:
\begin{verbatim}
cargo run --features tls -- --tls-cert cert.pem --tls-key key.pem
\end{verbatim}
Example starting a testnet node:
\begin{verbatim}
cargo run -- --network testnet --port 6001 --node-name node1
\end{verbatim}
To run multiple nodes locally, start each in its own terminal and list the other as a peer. The interactive prompt supports:
\begin{itemize}
\item \texttt{tx <recipient> <amount>} to send a transaction.
\item \texttt{add <peer\_addr>} to add a peer.
\item \texttt{remove <peer\_addr>} to drop a peer.
\item \texttt{list} to show current peers.
\item \texttt{balance} to display your wallet balance.
\item \texttt{puzzle-stats [<address>]} (testnet only) to show puzzle ownership and attempt counts.
\end{itemize}
Example puzzle queries:
\begin{verbatim}
puzzle-stats
puzzle-stats alice
\end{verbatim}

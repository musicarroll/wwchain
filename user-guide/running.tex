\chapter{Running a Node}
Launch a node with optional arguments:
\begin{verbatim}
cargo run -- --port <PORT> --node-name <NAME> --peers <PEER1,PEER2,...> --chain-dir <DIR> --network <mainnet|testnet>
\end{verbatim}
When the \texttt{tls} feature is enabled you can provide certificate paths:
\begin{verbatim}
cargo run --features tls -- --tls-cert cert.pem --tls-key key.pem
\end{verbatim}
Example starting a testnet node:
\begin{verbatim}
cargo run -- --network testnet --port 6001 --node-name node1
\end{verbatim}
To run multiple nodes locally, start each in its own terminal and list the other as a peer. The interactive prompt supports:
\begin{itemize}
\item \texttt{tx <recipient> <amount>} to send a transaction.
\item \texttt{add <peer\_addr>} to add a peer.
\item \texttt{remove <peer\_addr>} to drop a peer.
\item \texttt{list} to show current peers.
\item \texttt{balance} to display your wallet balance.
\item \texttt{search <regex>} to query recent blocks or transactions for matching text.
\item \texttt{puzzle-stats [<address>]} (testnet only) to show puzzle ownership and attempt counts.
\end{itemize}
Example puzzle queries:
\begin{verbatim}
puzzle-stats
puzzle-stats alice
\end{verbatim}

Example search queries:
\begin{verbatim}
search alice
search "bob|carol"
\end{verbatim}

Each transaction immediately mines a new block; the node does not create blocks on a periodic schedule.

\section{REST Shim and App Integration}
The project includes an optional HTTP REST shim that translates simple HTTP requests into signed network messages, so external apps (e.g., the Wordwalk Django site) can interact with the node without reimplementing the binary TCP protocol or signature logic.

\subsection{Starting the shim}
\begin{verbatim}
WWCHAIN_PEER_ADDR=127.0.0.1:6001 \\
WWCHAIN_REST_ADDR=127.0.0.1:7000 \\
# optional token auth
# WWCHAIN_REST_TOKEN=changeme \\
# optional fixed signing key (32 bytes hex). If unset, an ephemeral key is generated.
# WWCHAIN_SECRET_HEX=<64 hex chars> \\
cargo run --bin rest_shim
\end{verbatim}

\subsection{Endpoints}
\begin{itemize}
\item \texttt{GET /health} $\to$ JSON with \texttt{status}, \texttt{peer\_reachable}
\item \texttt{GET /info} $\to$ JSON \texttt{\{ pubkey\_hex, peer\_addr, network, height, peer\_reachable \}}
\item \texttt{POST /text} with \texttt{\{ "text": "hi", "peer": "127.0.0.1:6001"? \}} $\to$ signed \texttt{Text}
\item \texttt{POST /event} with \texttt{\{ "kind": "puzzle\_complete", "data": \{...\}, "peer"? \}} $\to$ signed \texttt{AppEvent}
\item \texttt{POST /tx} with \texttt{\{ "recipient": "<hex pub>", "amount": 1, "nonce": 42, "peer"? \}} $\to$ signed \texttt{Transaction}
\item \texttt{GET /balance/<pubkey>} $\to$ JSON \texttt{\{ pubkey, balance, height, next\_nonce \}}
\item \texttt{GET /chain} returns the full chain as JSON (list of blocks).
\item \textbf{Client-signed forwarding}:
  \begin{itemize}
  \item \texttt{POST /tx\_signed} with \texttt{\{ "signed": SignedMessage, "peer"? \}}; requires \texttt{Transaction} payload inside \texttt{SignedMessage}
  \item \texttt{POST /event\_signed} with \texttt{\{ "signed": SignedMessage, "peer"? \}}; requires \texttt{AppEvent} payload inside \texttt{SignedMessage}
  \end{itemize}
\item \textbf{Signature verification}: \texttt{POST /verify\_signature} with \texttt{\{ "pubkey": "<66 hex>", "message": "<utf8>", "signature": "<128 hex>" \}} $\to$ JSON \texttt{\{ valid: bool \}}
\end{itemize}

When \texttt{WWCHAIN\_REST\_TOKEN} is set the shim requires either an \texttt{X-Auth-Token} header or \texttt{Authorization: Bearer <token>} on POST requests (\texttt{/text}, \texttt{/event}, \texttt{/tx}).

\subsection{Client-signed flow}
For non-custodial clients (e.g., browser wallets) that hold their own keys:
\begin{enumerate}
\item Build a \texttt{NetworkMessage} (\texttt{Transaction} or \texttt{AppEvent}).
\item Wrap it in \texttt{VersionedMessage} (\texttt{\{version:1, payload\}}), serialize, and sign its SHA-256 digest with secp256k1 (64-byte compact), producing a \texttt{SignedMessage}.
\item Send the \texttt{SignedMessage} to \texttt{/tx\_signed} or \texttt{/event\_signed}.
\end{enumerate}
Use \texttt{/verify\_signature} to verify ECDSA signatures over arbitrary UTF-8 messages (e.g., one-time challenges) given a compressed secp256k1 public key.

\subsection{AppEvent structure}
For structured application messages the shim sends a signed \texttt{AppEvent}:
\begin{verbatim}
{ "version": 1, "payload": { "AppEvent": { "kind": "puzzle_complete",
  "data": { "attempt_id": 1, "puzzle_id": 2, "user_id": 3,
            "score": 50, "elapsed_ms": 12345, "ts": "2025-01-01T00:00:00Z",
            "commitment": "<sha256 of canonical data>" } } } }
\end{verbatim}

\subsection{Wordwalk (Django) integration}
The Wordwalk site uses the shim in development and testnet. Key points:
\begin{itemize}
\item Set \texttt{WWCHAIN\_REST\_URL} (e.g., \texttt{http://127.0.0.1:7000}) and optional \texttt{WWCHAIN\_REST\_TOKEN} in the Django environment.
\item A feature flag \texttt{CHAIN\_EVENTS\_ENABLED} guards all sends and defaults to off for safety.
\item A superuser-only test page at \texttt{/admin-tools/wwchain/} provides quick POST tests and shows recent send logs and the current nonce.
\item Puzzle completion emits an \texttt{AppEvent} (best-effort, with a hashed commitment of key fields); marketplace transactions send a signed \texttt{Transaction} plus a \texttt{trade} \texttt{AppEvent} when enabled.
\item Failed sends are queued in an outbox for retry; admin commands:
\begin{verbatim}
python manage.py flush_chain_outbox --max 50
python manage.py index_chain_events
python manage.py summarize_chain_events --period day --days 14 --source both
\end{verbatim}
\end{itemize}

For production deployments keep the shim bound to localhost, enable token auth, and leave \texttt{CHAIN\_EVENTS\_ENABLED} off until integration is planned.

\section{Chains, Difficulty, and Testnet}
Mainnet and testnet differ in difficulty targets. The node automatically lowers difficulty for testnet mining; the shim aligns validation during reconciliation so that \texttt{/info}, \texttt{/balance}, and \texttt{/chain} reflect the correct testnet chain.

\section{CLI Tips}
From the node prompt:
\begin{itemize}
\item \texttt{balance} prints this node's wallet balance.
\item \texttt{balance <pubkey>} prints the balance for an arbitrary address.
\item \texttt{tx <recipient> <amount>} mines immediately in a single-node setup.
\end{itemize}
